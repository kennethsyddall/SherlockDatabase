\documentclass{article}
\usepackage[utf8]{inputenc}
\title{Interception of Communications Act}
\author{}
\date{}
\begin{document}
\maketitle
Belize
Interception of Communications Act


 
Sections 3-7


 
Part II
Adoption Date:
2010-12-16


 
Original Text




3. Prohibition of interception
(1) Except as provided in this section, any person who with intent intecepts communication in the course of its transmission by means of a public postal service or a communication network without authorisation, commits an offence and, on conviction on indictment, is liable to:
(a) a fine of not less than twenty five thousand dollars and not exceeding fifty thousand dollars or to a term of imprisonment not exceeding three years in the first instance;
(b) a fine of not less than fifty thousand dollars and not exceeding one hundred thousand dollars or to a term of imprisonment not exceeding five years in the second instance; and 
(c) a fine of one hundred thousand dollars and a term of imprisonment not exceeding five years in the subsequent instances.
(a) a fine of not less than fifty thousand dollars and not exceeding one hundred thousand dollars or to a term of imprisonment not exceeding five years in the first instance;(b) a fine of not less than one hundred thousand dollars and not exceeding two hundred thousand dollars or to a term of imprisonment not exceeding ten years in the second instance; and
(c) a fine of two hundred thousand dollars and a term of imprisonment not exceeding ten years in the subsequent instances,
(3) A person does not commit an offence under subsection (1) if:
(a) the communication is intercepted in accordance with an interception direction issued pursuant to section 9;
(b) that person has written or otherwise documented authorisation consenting to the interception from the person to whom or from whom the communication is transmitted;
(c) the communication is acquired in accordance with the provisions of the Belize Telecommunications Limited Act, the Financial Intelligence Unit Act or or any other law; or
(d) the interception is of a communication made through a communication network that is so configured as to render the communication readily accessible to the general public.
(4) A court convicting a person of an offence under this Act may, in addition to any penalty which it imposes in respect of the offence, order the forfeiture or disposal of any device used in the commission of the offence as provided by section 26 (3) to (9).
(5) For the purposes of this section, communication means:
(a) communication in the course of transmission by means of a communication network in real time; or;
(b) communication stored in a manner that enables the recipient to receive that communication or otherwise have access to it.
4. Other interception related offences
A person who encrypts data of any description for the purpose of committing a crime or with intent utilises encrypted information for the purpose of committing a crime commits an offence and, on conviction on indictment, is liable to:
(a) a fine of not less than fifty thousand dollars and not exceeding one hundred thousand dollars or to a term of imprisonment not exceeding five years in the first instance;
(b) a fine of not less than one hundred thousand dollars and not exceeding two hundred thousand dollars or to a term of imprisonment not exceeding ten years in the second instance; and
(c) a fine of two hundred thousand dollars and a term of imprisonment not exceeding ten years in the subsequent instances.
5. Application for interception direction etc.
(1) An authorised officer who wishes to obtain an interception direction pursuant to the provisions of this Act shall request the Director of Public Prosecutions to make an application ex parte to a Judge in chambers on his behalf.
(2) An application referred to in subsection (1) shall be in the prescribed form and shall be accompanied by an affidavit deposing the following:
\documentclass{article}
\usepackage[utf8]{inputenc}
\title{Interception of Communications Act}
\author{}
\date{}
\begin{document}
\maketitle
Belize
Interception of Communications Act


 
Sections 3-7


 
Part II
Adoption Date:
2010-12-16


 
Original Text




3. Prohibition of interception
(1) Except as provided in this section, any person who with intent intecepts communication in the course of its transmission by means of a public postal service or a communication network without authorisation, commits an offence and, on conviction on indictment, is liable to:
(a) a fine of not less than twenty five thousand dollars and not exceeding fifty thousand dollars or to a term of imprisonment not exceeding three years in the first instance;
(b) a fine of not less than fifty thousand dollars and not exceeding one hundred thousand dollars or to a term of imprisonment not exceeding five years in the second instance; and 
(c) a fine of one hundred thousand dollars and a term of imprisonment not exceeding five years in the subsequent instances.
(2) A person who with intent intercepts a communication in the course of its transmission by means of a public postal service or a communication network for the purpose of commercial benefit, political advantage', or criminal activity commits an offence and, on conviction on indictment, is liable to:
(a) a fine of not less than fifty thousand dollars and not exceeding one hundred thousand dollars or to a term of imprisonment not exceeding five years in the first instance;(b) a fine of not less than one hundred thousand dollars and not exceeding two hundred thousand dollars or to a term of imprisonment not exceeding ten years in the second instance; and
(c) a fine of two hundred thousand dollars and a term of imprisonment not exceeding ten years in the subsequent instances,
(3) A person does not commit an offence under subsection (1) if:
(a) the communication is intercepted in accordance with an interception direction issued pursuant to section 9;
(b) that person has written or otherwise documented authorisation consenting to the interception from the person to whom or from whom the communication is transmitted;
(c) the communication is acquired in accordance with the provisions of the Belize Telecommunications Limited Act, the Financial Intelligence Unit Act or or any other law; or
(d) the interception is of a communication made through a communication network that is so configured as to render the communication readily accessible to the general public.
(4) A court convicting a person of an offence under this Act may, in addition to any penalty which it imposes in respect of the offence, order the forfeiture or disposal of any device used in the commission of the offence as provided by section 26 (3) to (9).
(5) For the purposes of this section, communication means:
(a) communication in the course of transmission by means of a communication network in real time; or;
(b) communication stored in a manner that enables the recipient to receive that communication or otherwise have access to it.
4. Other interception related offences
A person who encrypts data of any description for the purpose of committing a crime or with intent utilises encrypted information for the purpose of committing a crime commits an offence and, on conviction on indictment, is liable to:
(a) a fine of not less than fifty thousand dollars and not exceeding one hundred thousand dollars or to a term of imprisonment not exceeding five years in the first instance;
(b) a fine of not less than one hundred thousand dollars and not exceeding two hundred thousand dollars or to a term of imprisonment not exceeding ten years in the second instance; and
(c) a fine of two hundred thousand dollars and a term of imprisonment not exceeding ten years in the subsequent instances.
5. Application for interception direction etc.
(1) An authorised officer who wishes to obtain an interception direction pursuant to the provisions of this Act shall request the Director of Public Prosecutions to make an application ex parte to a Judge in chambers on his behalf.
(2) An application referred to in subsection (1) shall be in the prescribed form and shall be accompanied by an affidavit deposing the following:
(a) the name of the authorised officer on behalf of whom the application is made;
(b) the facts or allegations giving rise to the application;
(c) sufficient information for a Judge to issue an interception direction;
(d) the ground referred to in section 6(1) on which the application is made;
(e) full particulars of all the facts and the circumstances alleged by the authorised officer on whose behalf the application is made including:
(i) if practical, a description of the nature and location of the facilities from which or the premises at which the communication is to be intercepted; and 
(ii) the basis for believing that evidence relating to the ground on which the application is made will be obtained through the interception;
(f) if applicable, whether other investigative procedures have been applied and failed to produce the required evidence or the reason why other investigative procedures reasonably appear to be unlikely to succeed if applied, or are likely to be too dangerous to apply in order to obtain the required evidence;
(g) the period for which the interception direction is required to be issued;
(h) whether any previous application has been made for the issuing of an interception direction in respect of the same person, the direction in respect of the same person, the same facility or the same premises specified in the application and, if such previous application exists, shall indicate the current status of that application; and
(i) if applicable, a description of the communication equipment to be intercepted;
(j) any other directives issued by the Judge.
(3) Subsection (2)(d) shall not apply in respect of an application for the issuing of an interception direction on a ground referred to in section 6(1) (a) if a serious offence has been or is being or will probably be committed for the benefit of, or at the directions of, or in association with, a person, a group of persons or syndicate involved in organised crime or groups classified as criminal gangs.
(4) Where an interception direction is applied for on the grounds of national security, the application from the authorised officer shall be accompanied by a written authorisation signed by the Minister authorising the application on that ground.
(5) Subject to subsection (6), the records relating to an application for an interception direction or the renewal or modification thereof shall be:
(a) placed in a packet and sealed by the Judge to whom the application is made immediately on determination of the application; and
(b) kept in the custody of the court in a place to which the public has no access or such a place as the Judge may authorise.
(6) The records referred to in subsection (5) may be opened if a Judge so orders only:
(a) for the purpose of dealing with an application for further authorisation; or
(b) for renewal of an authorisation.
6. Issuance of an interception direction
(1) An interception direction shall be4 issued if a Judge is satisfied, on the facts alleged in the application pursuant to section 5, that there are reasonable grounds to believe that:
(a) obtaining the information sought under the interception direction is necessary in the interest of:
(i) national security;
(ii) public order;
(iii) public safety; or
(iv) public health;
(v) preventing detecting, investigating, or prosecuting any offence specified in the Schedule where there are reasonable grounds to believe that such an offence has been, is being or may be committed; or
(vi) giving effect to the provisions of any mutual legal assistance agreement in circumstances appearing to the Judge to be equivalent to those in which he would issue ab interception direction by virtue of sub-paragraph (v); and
(b) other procedures:
(i) have not been or are unlikely to be successful in obtaining the information sought to be acquired by means of the interception direction;
(ii) are too dangerous to adopt in the circumstances; and
(c) it would be in the best interests of the administration of justice to issue the interception direction.
(2) A Judge considering an application may require the authorised officer to furnish such further information as he deems necessary.
7. Scope and form of application
(1) An interception direction shall be in the prescribed form and shall permit the authorised officer to:
(a) intercept, at any place in Belize, any communication in the course of its transmission;
(b) secure the interception in the course of its transmission by means of postal service or a public or private communication network, of such communication as are described in the interception direction; and
(c) secure the disclosure of the intercepted material obtained or required by the interception direction, and of related communication data.
(2) An interception direction shall authorise the interception of:
(a) communication transmitted by means of a postal service or a public or private communication network to or from:
(i) the person specified in the interception direction;
(ii) the premises so specified and described; or
(iii) the set of communication, if any as may be necessary in order to intercept communication falling within paragraph (a).
(3) An interception direction shall specify the identity of the:
(a) authorised officer on whose behalf the application is made pursuant to section 5, and the person who will execute the interception direction;
(b) person, if known and appropriate, whose communication is to be intercepted; and
(c) postal service provider or the communication provider to whom the interception direction to intercept must be addressed, if applicable.
(4) An interception direction may contain such ancillary provisions as are necessary to secure its implementation in accordance with the provisions of this Act.
(5) An interception direction issued pursuant to this section may specify conditions or restrictions relating to the interception of communications authorised therein.






 


 
Cross-Cutting Issues
Liability
Liability of Legal Persons

                          • Criminal
                        
International Cooperation: 

                          • Electronic Evidence/Digital Evidence
                        


 
Attachments

                                         Interception of Communications Act
                                    
Comment




SCHEDULE 




INTERCEPTION OF COMMUNICATIONS ACT 




Communications that are subject to Interception [Section 6 (1)(a) (vi)]




1. Arson




2. Blackmail and extortion




3. Burglary




4. Corruption and illicit enrichment




5. Counterfeiting




6. Drug trafficking




7. Espionage




8. Forgery




9. Fraud




10. Hijacking




11. Human smuggling




12. Kidnapping or abduction




13. Manslaughter




14. Membership in a criminal gang




15. Money laundering




16. Murder




17. Rape, sexual assault, unlawful carnal knowledge




18. Robbery




19. Subversion




20. Tax evasion




21. Terrorism




22. Theft




23. Threatening and intimidation




24. Trafficking in persons




25. Treason




26. Unlicensed and prohibited explosives, weapons, and ammunition, etc.




27. Attempting or conspiring to commit, aiding, abetting, counselling, or procuring the commission of, an offence falling within any of the preceding paragraphs




\end{document}
